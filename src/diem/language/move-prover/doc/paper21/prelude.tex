
\usepackage{xcolor}
\usepackage{listings}
\usepackage{inconsolata}
\usepackage{relsize}
%\usepackage{amsmath,amssymb,array}
\usepackage{mathpartir}
\usepackage{graphicx}
\usepackage{xspace}
\usepackage{breakurl}
\usepackage[breaklinks,colorlinks=true]{hyperref}
\def\UrlBreaks{\do\/\do-}
\usepackage{caption}

\newcommand{\Section}[1]{\vspace{-0ex}\section{#1}}
\newcommand{\SubSection}[1]{\vspace{-0ex}\subsection{#1}}
\newcommand{\Paragraph}[1]{\vspace{-1.0ex}\subsubsection{#1}} % lncs

\newcommand{\TODO}[2]{%
  \noindent\textcolor{gray}{%
      {\scriptsize\textbf{TODO(#1)}: #2}
  }
}

\setlength{\belowcaptionskip}{-1ex}
\setlength{\textfloatsep}{0.2ex}
\newenvironment{Figure}{
  \begin{figure}[tb]
} {
  \end{figure}
}

\newcommand{\MVP}[0]{\textsf{MVP}\xspace}
\newcommand{\solidity}[0]{\textsf{Solidity}\xspace}
\newcommand{\kay}[0]{\textsf{K}\xspace}
\newcommand{\fstar}[0]{\textsf{f*}\xspace}



% Source Transformation

\newcommand{\transform}[0]{\Large{$\leadsto$}}


%%%% Code

\usepackage[charter]{mathdesign}
%\def\rmdefault{bch}
%\def\ttdefault{blg}

\lstdefinestyle{MoveStyle}{
  basicstyle=\ttfamily,
  keywordstyle=\color{black}, % don't bold keywords which is the default
  escapechar=@, % use to embed LaTeX into code
  breaklines=true,
}


\lstdefinelanguage{Move}{
  morekeywords={
    abort,
    aborts_if,
    acquires,
    address,
    as,
    assert,
    assume,
    borrow_global,
    borrow_global_mut,
    break,
    const,
    continue,
    copy,
    copyable,
    define,
    drop,
    else,
    ensures,
    exists,
    false,
    forall,
    friend,
    fun,
    global,
    has,
    havoc,
    if,
    include,
    invariant,
    key,
    let,
    loop,
    modifies,
    module,
    move,
    move_from,
    move_to,
    mut,
    native,
    num,
    old,
    onabort,
    pragma,
    public,
    requires,
    resource,
    return,
    schema,
    script,
    signer,
    spec,
    store,
    struct,
    true,
    u8,
    u64,
    u128,
    update,
    use,
    with,
    where,
    while},
  sensitive=true,
  morecomment=[l]{//},
  morecomment=[s]{/*}{*/},
}

% This allows us to use |<move code>| for inline code.
\lstMakeShortInline[language=Move,style=MoveStyle]|


% This defines a new environment for Move code.
\lstnewenvironment{Move}{
  \lstset{
    language=Move,
    style=MoveStyle,
    basicstyle=\relsize{-1}\ttfamily,
    keywordstyle=\color{blue},
  }
}{
}

% This defines a new environment for Move code in a box, without line numbers.
\lstnewenvironment{MoveBox}{
  \lstset{
    language=Move,
    style=MoveStyle,
    basicstyle=\relsize{-1}\ttfamily,
    keywordstyle=\color{blue},
    %numbers=left,
    %numberstyle=\scriptsize\color{gray},
    %frame=single,
  }
}{
}

% This defines a new environment for Move code in a box, with line numbers.
\lstnewenvironment{MoveBoxNumbered}{
  \lstset{
    language=Move,
    style=MoveStyle,
    basicstyle=\relsize{-1}\ttfamily,
    keywordstyle=\color{blue},
    numbers=left,
    numberstyle=\scriptsize\color{gray},
    %frame=single,
  }
}{
}

% This defines a new environment for diagnostics as produced by the prover.
\lstnewenvironment{MoveDiag}{
  \lstset{
    style=MoveStyle,
    basicstyle=\relsize{-2}\ttfamily,
  }
}{
}




%%% Local Variables:
%%% mode: latex
%%% TeX-master: "main"
%%% End:
