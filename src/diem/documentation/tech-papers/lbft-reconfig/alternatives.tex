\section{Alternatives} \label{alternatives}
In this section we briefly discuss some other approaches we considered for the problem and the trade offs.

\subsection{Quorum Compatible}
As we mentioned in \ref{safety}, we give up the pipeline of \LBFT to preserve safety. Here we consider an alternative that preserves
both safety and pipeline.

\paragraph{Condition}

Consider a reconfiguration transaction that changes from validators \rust{N1} to validators \rust{N2} and $f1=\lceil|N1|/3\rceil-1$,
$f2=\lceil|N2|/3\rceil-1$ malicious power, $|N1|-f1$,  $|N2|-f2$ as quorum size respectively. We only allow reconfiguration if

\begin{equation}
|N1 \cap N2| > f1 + f2 + max(f1, f2)
\end{equation}

\paragraph{Analysis}
We have an important lemma in \LBFT for safety: Under BFT assumption, for every two quorums of nodes, there exists an honest node that
belongs to both quorums.

With this approach, we restrict the changes between two configurations and we can prove quorums are interchangeable in both configurations,
so that we implicitly maintain the safety property.

\subsection{Pipelined Configuration}
This is one step further on top of the quorum compatible approach. Instead of implicit compatible quorums, we require explicit two quorums
to commit the reconfiguration transaction. We preserve the same safety lemma as above but doing it in a more explicit way.

One way to think about this is pipelining the reconfiguration like we pipeline blocks, reconfiguration $A\rightarrow B$ takes effect
(the next blocks on this branch need B’s quorum) right after the block and until a commit of another reconfiguration $B\rightarrow X$.

\subsection{Comparison}
We assemble a comparison table for the three approaches we describe.
\begin{table*}[h]
\begin{tabular}{|c|c|c|c|}

\hline
& Pipelining & Implementation Complexity & Flexibility\\
\hline \hline
Basic & No & Small & Most\\
Quorum Compatible & Yes & Medium & Least\\
Pipelined Configuration & Yes & Big & Medium\\
\hline

\end{tabular}
\end{table*}

Basic mechanim gives up pipelining/performance while provides more flexible configuration changes like protocol change, it also provides better
isolation between different configurations.

Quorum Compatible enables us with pipelining and is almost seamless to \LBFT by strictly limit how configuration can change.

Pipelined configuration provides pipelining the same as Quorum Compatible with more flexible reconfiguration by supporting explicit multi quorums.

Also regardless of implicit or explicit quorums, to support pipelining configuration, we need to track configuraiton per block tree branch which
adds complexity compared to one configuration per whole block tree in basic mechanism.

We chose the approach considered all above and think the performance gain of pipelining can not justify the added complexity and
lost flexibility for now, but could leave for future work.
